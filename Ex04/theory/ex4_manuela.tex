\documentclass{article}

% formatting and layout
\usepackage[left=2.5cm, right=2.5cm, bottom=2.5cm]{geometry}
\usepackage[onehalfspacing]{setspace}
\setlength{\parindent}{0pt}

% input/output language
\usepackage[utf8]{inputenc}
\usepackage[T1]{fontenc}
\usepackage[ngerman]{babel}

% math packages
\usepackage{amsmath, amsfonts, dsfont, amsthm}
\usepackage{tikz}

% new commands
\newcommand{\Prob}[1]{\mathbb{P}\left(#1\right)}
\newcommand{\prob}[1]{p\!\left(#1\right)}
\newcommand{\cProb}[2]{\mathbb{P}\left(#1 \mid #2\right)}
\newcommand{\cprob}[2]{p\left(#1 \mid #2\right)}
\newcommand{\Bi}[3]{\text{Bi}_{{#1}, {#2}}\left( #3 \right)}
\newcommand{\exponential}[1]{e^{#1}}
\newcommand{\N}{\mathcal{N}}
\newcommand{\expect}[1]{\mathrm{E}\left[#1\right]}
\newcommand{\var}[1]{\mathrm{Var}\left[#1\right]}
\newcommand{\dx}{\mathrm{d}x}
\newcommand{\dy}{\mathrm{d}y}
\newcommand{\lagr}{\mathcal{L}}
\newcommand{\lagrange}[1]{\mathcal{L}\left( #1 \right)}
\newcommand{\G}{\mathcal{G}}
\newcommand{\real}{\mathbb{R}}

\begin{document}
\begin{center}
	\Huge \textsc{Machine Learning 2} \\
    \Large \textbf{Exercise Sheet 4}
\end{center}

\hrule

\subsection*{Exercise 1: Convolution Kernel} 
\begin{itemize}
\item[\textbf{(a)}]
Let $c_1, \dots, c_N \in \real$. For inputs $x_1,\dots,x_N$ we then have
\begin{align*}
\sum_{i=1}^{N} \sum_{j=1}^{N} c_i c_j k(x_i, x_j)
	&= \sum_{i=1}^{N} \sum_{j=1}^{N} c_i c_j \sum_{t=-\infty}^{\infty} \left( [ x_i * x_j ]_t \right)^2 \\
	&= \sum_{i=1}^{N} \sum_{j=1}^{N} c_i c_j \sum_{t=-\infty}^{\infty} \left( \sum_{\tau=-\infty}^{\infty} x_i(\tau) \cdot x_j(t - \tau) \right)^2 \\
	&= \sum_{j=1}^{N} c_j \sum_{i=1}^{N} c_i \sum_{t=-\infty}^{\infty} \left( \sum_{\tau=-\infty}^{\infty} x_i(\tau) \cdot x_j(t - \tau) \right)^2 \\
	&= \sum_{j=1}^{N} c_j \sum_{t=-\infty}^{\infty} \left( \sum_{\tau=-\infty}^{\infty} \sum_{i=1}^{N} c_i x_i(\tau) \cdot x_j(t - \tau) \right)^2 \\
	&= \sum_{t=-\infty}^{\infty} \left( \sum_{\tau=-\infty}^{\infty} \sum_{i=1}^{N} c_i x_i(\tau) \cdot \sum_{j=1}^{N} c_j x_j(t - \tau) \right)^2 \\
	&\geq 0
\end{align*}
\item[\textbf{(b)}]
\end{itemize}

\subsection*{Exercise 2: Weighted Degree Kernels} 
\begin{itemize}
\item[\textbf{(a)}]
\item[\textbf{(b)}]
\item[\textbf{(c)}]
\end{itemize}

\section*{Exercise 3: Fisher Kernels} 
\begin{itemize}
\item[\textbf{(a)}]
\item[\textbf{(b)}]
\end{itemize}
\end{document}
